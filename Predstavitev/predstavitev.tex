\documentclass{beamer}
\usepackage[utf8]{inputenc} % šumniki
\usepackage[slovene]{babel} % preklopimo na slovenščino
\usepackage{amsmath}
\usepackage{amsfonts}
%\usepackage{theorem}
\usepackage{amsthm}
\usepackage{ wasysym }
\usepackage{listings}
\usepackage{color}
\definecolor{viola}{RGB}{236,236,255}
\lstset{ %
  backgroundcolor=\color{viola},   % choose the background color; you must add \usepackage{color} or \usepackage{xcolor}
  basicstyle=\footnotesize,        % the size of the fonts that are used for the code
  breakatwhitespace=false,         % sets if automatic breaks should only happen at whitespace
  breaklines=true,                 % sets automatic line breaking
  %captionpos=b,                    % sets the caption-position to bottom
  %commentstyle=\color{mygreen},    % comment style
  %deletekeywords={},            % if you want to delete keywords from the given language
  %escapechar=@,          % if you want to add LaTeX within your code
  extendedchars=true,              % lets you use non-ASCII characters; for 8-bits encodings only, does not work with UTF-8
  frame=single,                    % adds a frame around the code
  keepspaces=true,                 % keeps spaces in text, useful for keeping indentation of code (possibly needs columns=flexible)
  keywordstyle=\color{blue},       % keyword style
  language=Haskell,                 % the language of the code
  %morekeywords={*,...},            % if you want to add more keywords to the set
  numbers=none,                    % where to put the line-numbers; possible values are (none, left, right)
  %numbersep=5pt,                   % how far the line-numbers are from the code
  %numberstyle=\tiny\color{mygray}, % the style that is used for the line-numbers
  %rulecolor=\color{black},         % if not set, the frame-color may be changed on line-breaks within not-black text (e.g. comments (green here))
  showspaces=false,                % show spaces everywhere adding particular underscores; it overrides 'showstringspaces'
  showstringspaces=false,          % underline spaces within strings only
  showtabs=false,                  % show tabs within strings adding particular underscores
  %stepnumber=2,                    % the step between two line-numbers. If it's 1, each line will be numbered
  %stringstyle=\color{mymauve},     % string literal style
  tabsize=4,                       % sets default tabsize to 2 spaces
  %title=\lstname                   % show the filename of files included with \lstinputlisting; also try caption instead of title
}

\usetheme{Warsaw}

\useoutertheme{infolines}
\author[Sintaktična sladkorčka]{Tomaž Stepišnik Perdih\\  Matej Petković}

%\institute[]{FMF}
\title{Izbire}
\date{{22. januar 2015}}
%\institute[]{}

\newtheorem{defin}{Definicija}
\newtheorem{vpr}{Vprašanji}
\newtheorem{odg}{Odgovora}
\newtheorem{dok}{Vprašanje}
\newtheorem{posl}{Posledica}
\newtheorem{koda}{Minuta za Haskell}
\newtheorem{prim}{Primer}
\def\Rn{\mathbb{R}^n}
\def\izo{\approx}
\def\eps{\varepsilon}
\def\obs{\exists}
\def\fora{\forall}
\newcommand{\sgn}{\operatorname{sgn}}
\newcommand{\df}{\operatorname{df}}
\newcommand{\argmin}{\operatorname{argmin}}
\newcommand{\argmax}{\operatorname{argmax}}
\newcommand{\maxi}{\operatorname{maxi}}
\newcommand{\mini}{\operatorname{mini}}


\begin{document}

\begin{frame}
\titlepage 
\end{frame}

\begin{frame}{Uvodni pojmi}
Imejmo neprazno množico $A$, v kateri so elementi tipa \texttt{a}  in množico logičnih vrednosti $r =  \{ \texttt{True}, \texttt{False}\} $. Definiramo lahko
\begin{itemize}
\item predikate\; \lstinline!p :: a -> r!
\item kvantifikatorje\; \lstinline[mathescape]! $\phi$ :: (a -> r) -> r![
\item izbire\; \lstinline[mathescape]! $\eps$  :: (a -> r) -> a![
\end{itemize}

\pause

\begin{prim}
Naj bo $A = \{1,2,3,4\}$ in $\phi = \exists_A$. Potem je s predpisom
$$\eps (p) =
\begin{cases}
\min \{ n \in A \mid p(n) \} ;& \phi(p)\\
\hphantom{oooooooooioooo}1\, ;& sicer
\end{cases}
$$
definirana izbira za eksistenčni kvantifikator za množico $A$. Torej je $\eps (n \to n\%2 == 0) = 2$ in $\eps (n \to n > 4) = 1$.
\end{prim}

\end{frame}

\begin{frame}[fragile]{Izbira splošneje}%[fragile] zaradi verbatima

\begin{dok}
Kakšna je izbira za univerzalni kvantifikator?
\end{dok}

\pause

Če je $\eps$ izbira za univerzalni kvantifikator $\phi(p) = \forall x \in A . p(x)$, potem velja:
\begin{itemize}
\item $\eps(p) \in A$,
\item če velja $p(\eps(p))$, potem velja $p(x)$ za vse $x \in A$.
\end{itemize}


\end{frame}

\begin{frame}[fragile]{Posplošeni predikati}

Denimo, da imamo prodajalno omar. Lahko definiramo predikat, ki vsaki omari določi ceno \lstinline$p :: Omara -> Cena$. 
\begin{itemize}
\item Najnižjo in najvišjo ceno omare lahko potem najdemo s kvantifikatorjema
\begin{lstlisting}
maxi,mini :: (Omara -> Cena) -> Cena
\end{lstlisting}
\item Najcenejšo in najdražjo omaro pa najdemo z izbirama
\begin{lstlisting}
argmin,argmax :: (Omara -> Cena) -> Omara
\end{lstlisting}
\end{itemize}
\end{frame}

\begin{frame}{Križci in krožci}
Matej in Tomaž igrata križce in krožce. Označimo z $A_n$ množico prostih polj na $n$-ti potezi, $1\leq n\leq 9$.
Označimo Matejeve poteze z $x_i$, Tomaževe pa z $y_i$. Denimo, da začne Matej.  Zanima ga, ali velja
$$\obs x_1\in A_1.\fora y_2 \in A_2. \obs x_3\in A_3.\fora y_4 \cdots \obs x_9 \in A_9. \operatorname{zmaga}_\text{M}(x_1,y_2,\dots, y_8,x_9) \text{,}$$
\pause
kar lahko, če Matejevo zmago označimo z $1$, Tomaževo z $-1$ ter izenačenje z $0$, prepišemo v

$$\max_{x_1}. \min_{y_2}. \cdots \max_{x_9}. \operatorname{vrednost}(x_1, y_2,\dots, y_8, x_9) = 1\text{.}$$


\end{frame}

\begin{frame}
\begin{vpr}
	\begin{itemize}
	\item Kako se končajo križci in krožci (ter podobne igre), če oba igralca igrata optimalno? (kvantifikatorji)
	\item Kakšno je optimalno zaporedje potez? (izbire)
	\end{itemize}
\end{vpr}
\pause
\begin{odg}
	\begin{itemize}
	\item Izračunamo $$\max_{x_1}\min_{y_2}\cdots \max_{x_9}\;\operatorname{vrednost}(x_1, y_2,\dots, x_9)$$
	\item Za kvantifikatorja $\min$ in $\max$ najdemo pripadajoči izbiri $\argmin$ in $\argmax$. Potem vseh devet izbir
	za posamezno potezo zdužimo v izbiro zaporedja potez.
	\end{itemize}
\end{odg}

\end{frame}

\begin{frame}[fragile]{Združevanje izbir}%[fragile] zaradi verbatima
Združevanje izbir se malo zaplete, ker je vsaka poteza odvisna od vseh prejšnjih (tudi prva \smiley).
Označimo tip potez s \texttt{pot} in tip elementov naše množice $R$ z \texttt{r}.
\pause
\begin{lstlisting}
type J x = (x -> r) -> x
krat :: J pot -> (pot -> J [pot]) -> J [pot]
krat e0 e1 p = a0 : a1
    where
        a0 = e0(\x0 -> preveri (e1 x0) (\x1 -> p (x0 : x1)))
        a1 = e1 a0 (\x1 -> p(a0 : x1))
\end{lstlisting}
\pause
\begin{lstlisting}
prod :: [[x] -> J r x] -> J r [x]
prod [] = \p -> []
prod (e : es) =  
    (e []) `krat` (\x -> prod (map (\e xs -> e(x:xs)) es))
\end{lstlisting}
\end{frame}











\end{document}